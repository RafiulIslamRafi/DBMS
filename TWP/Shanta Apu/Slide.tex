\documentclass{beamer}

\mode<presentation>
{
  \usetheme{default}      % or try Darmstadt, Madrid, Warsaw, ...
  \usecolortheme{default} % or try albatross, beaver, crane, ...
  \usefonttheme{default}  % or try serif, structurebold, ...
  \setbeamertemplate{navigation symbols}{}
  \setbeamertemplate{caption}[numbered]
} 

\usepackage[english]{babel}
\usepackage[utf8x]{inputenc}
\usepackage{graphicx}

\title[Your Short Title]{Presentation on LED}
\author{Present by\\ Kanis Fatema}
\institute{\color{red}Department of Computer Science and Enigineering\\
	\color{green}\vspace{0.25cm}JATIYA KABI KAZI NAZRUL ISLAM UNIVERSITY}
\date{Date :24-12-18}

\begin{document}

\begin{frame}
  \titlepage
\end{frame}

% Uncomment these lines for an automatically generated outline.
%\begin{frame}{Outline}
%  \tableofcontents
%\end{frame}

\section{Overview}

\begin{frame}{Overview of the topic}

\begin{itemize}
  \item Introduction.
  \item Construction.
  \item working principle.
  \item Applications.
  \item Examples.
\end{itemize}

\vskip 1cm

%\begin{block}{Examples}
%Some examples of commonly used commands and features are %included, to help you get started.
%\end{block}

\end{frame}

\section{Introduction}

\subsection{Introduction of LED}

\begin{frame}{Introduction of LED}
\begin{center}
	A light-emitting diode (LED) is a semiconductor light source that emits light when current flows through it.
\end{center}

% Commands to include a figure:
%\begin{figure}
%	\includegraphics[width=0.15cm\linewidth, height=0.15\textheight]{../Desktop/ALGO/800px-LED,_5mm,_green_(en).svg}
	%\caption{}
	%\label{fig:800px-led5mmgreenen}
%\end{figure}

\end{frame}

\subsection{Construction}

\begin{frame}{Constructions}

Most LEDs were made in the very common 5 mm T1¾ and 3 mm T1 packages, but with rising power output, it has grown increasingly necessary to shed excess heat to maintain reliability,[39] so more complex packages have been adapted for efficient heat dissipation. Packages for state-of-the-art high-power LEDs bear little resemblance to early LEDs.
\end{frame}
\subsection{Working Principle}
\begin{frame}{Working Principle}
A light-emitting diode (LED) is a semiconductor light source that emits light when current flows through it. [5] When a suitable current is applied to the leads,[6][7] electrons are able to recombine with electron holes within the device, releasing energy in the form of photons. This effect is called electro luminescence. The color of the light (corresponding to the energy of the photon) is determined by the energy band gap of the semiconductor. White light is obtained by using multiple semiconductors or a layer of light emitting phosphor on the semiconductor device.
\end{frame}
\subsection{Applications}
\begin{frame}{Applications}
\begin{itemize}
	\item As General purpose lighting luminaries (including LED).
	\item Electric and electric hybrid vehicles.
	\item Appliances.
	\item Electrical devices.
\end{itemize}
\end{frame}
\subsection{Examples}
\begin{frame}{Examples of some LED}
	\begin{itemize}
		\item{Blue LED}
		\item{Red LED}
		\item{White LED}
		\item{Ultra-violate LED}
		\item{Violate LED}
		\item{Infrared LED}
	\end{itemize}
\end{frame}
\begin{frame}
\centering
	\color{blue}Thanks
\end{frame}

\end{document}