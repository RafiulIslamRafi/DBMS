\documentclass[12pt,a4paper]{article}
\usepackage[top=1in,bottom=1in,left=1in,right=1in]{geometry}
\usepackage{graphicx}
\graphicspath{{./pics/}}
\usepackage{listings}
\usepackage[utf8]{inputenc}

	
\begin{document}
	
		\begin{center}
		\LARGE\textbf{Jatiya Kabi Kazi Nazrul Islam University} \\
		\Large\textbf{Trishal, Mymensing-2220}\\
		\vspace{5mm}
		\includegraphics[width=3cm,height=3cm]{un.jpg}
			\end{center}

		\begin{center}	
			\large\textbf{Department Of Computer Science And Engineering}
			\textbf{\\Course Name: }Technical Writing\\
			\textbf{Course Code: }GED-272\\	
					{A Lab report on "Technical Writing"}\\
	\end{center}
\vspace{2cm}
\begin{flushleft}
\textbf{Submitted  To:}\\
Jannatul Ferdous\\
Associate Professor\\
Dept. of CSE\\
JKKNIU
\end{flushleft}

\begin{flushright}
\textbf{Submitted  by:}\\
 Mst. Sabrina Sultana\\
Reg. No: 5722\\
Roll: 17102045\\
Session : 2016-17\\
Dept. of CSE\\
\end{flushright}
\thispagestyle{empty}

\newpage
\begin{center}
	\large\textbf{Index}
\end{center}

\begin{tabular}{|p{1.5cm}|p{12.5cm}|p{1.5cm}|}
	\hline
	Exp.No& Experiment Name&Page No\\
	\hline
	01&To demonstrate a basic Latex document&1\\
	\hline
	02&To create a table using LaTex&2-3\\
	\hline
	03&To import an image in a LaTex document&4-5\\
	\hline
	04&To display mathematical equation using LaTex&6-7\\
	\hline
	05&To display  matrix using LaTex&8-9\\
	\hline
	06&To demonstrate a cv with cover letter using LaTex&10-14\\
	\hline
	07&To demonstrate a project report using LaTex&15-19\\
	\hline
	08&To demonstrate a lab report using LaTex&20-22\\
	\hline
	09&To demonstrate a thesis paper using LaTex&23-30\\
	\hline
		10&To demonstrate a project proposal using LaTex&31-33\\
	\hline
		11&To demonstrate a slide using LaTex&34-35\\
	\hline
	
\end{tabular}

\thispagestyle{empty}
\setcounter{page}{0}
%firstONE
\newpage
\begin{flushleft}
	

\textbf{Experiment No:} 01\\

\textbf{\\Experiment Name:} To demonstrate a basic Latex document.\\
\textbf{\\Objective:} To know how to create a PDF using Latex and how to use packages.\\
\textbf{\\Introduction:} LaTex is a typesetting system that can be used to create documents with proper formatting and notations. In a simple Latex document the font size,type and paper and also document type is defined in documentclass and a package called geometry is used to define the margin.\\
\textbf{\\Code:}
\begin{lstlisting}
\documentclass[12pt, a4paper, TimesNewRoman]{article}
\usepackage[left = 1in, right = 1in, top = 1in, bottom = 1in]{geometry}
\begin{document}
This is my first latex document which is
written in a text editor and given the output in pdf formate.
\end{document}
\end{lstlisting}
\textbf{\\Output:}
\includegraphics[width=17cm]{basic}\\
\textbf{\\Discussion:} A document of PDF format is created containing the written text within the document.
\end{flushleft}


%secondONE
\newpage
\begin{flushleft}
	
	
	\textbf{Experiment No:} 02\\
	\textbf{\\Experiment Name:} To create a table using LaTex.\\
	\textbf{\\Objective:} To know how to create a table in a Latex document.\\
	\textbf{\\Introduction:} LaTex is a typesetting system that can be used to create documents with proper formatting and notations. The notation {tabular} can be used to create a table using LaTex.\\
	\textbf{\\Code:}
\begin{lstlisting}
\documentclass[12pt,a4paper,TimesNewRoman]{article}
\usepackage[left=1in,right=1in,top=1in,bottom=1in]{geometry}
\begin{document}
\large\textbf{Table}\\
\begin{tabular}{|c|c|c|}
\hline
Name&Dept.&ID\\
\hline
Mitu&CSE&201\\
\hline
Tania Sultana Bristy&EEE&234\\
\hline
\end{tabular}
\large\textbf{Table with fixed column length}\\
\begin{tabular}{|p{4cm}|p{3cm}|p{1cm}|}
\hline
Name&Dept.&ID\\
\hline
Tania Sultana Bristy&CSE&201\\
\hline
Mili&EEE&234\\
\hline
\end{tabular}
\large\textbf{Table combaining multicolumn}\\
\begin{tabular}{|c|c|c|c|} \hline
\multicolumn{4}{|c|}{Students and their respective subjects} \\
\hline
Mily & Electrical & Geography & Mechanical  \\
\hline
Joya & ICT & English & Civil \\ 
\hline
Pinky & CSE & APECE & Botany  \\ 
\hline
Mitu & ECE & EEE & Biochemistry  \\ 
\hline
\end{tabular}

\end{document}
\end{lstlisting}
	\textbf{Output:}\\
	\includegraphics[height=8cm,width=5cm]{sec2}\\
	\textbf{\\Discussion:}  Table with different attribute is created successfully.
\end{flushleft}


%thirdONE
\newpage
\begin{flushleft}
	
	
	\textbf{Experiment No:} 03\\
	\textbf{\\Experiment Name:} To import an image in a LaTex document.\\
	\textbf{\\Objective:} To know how to import an image in a Latex document.\\
	\textbf{\\Introduction:} LaTex is a typesetting system that can be used to create documents with proper formatting and notations. Images are frequently used in every kind of work. LaTex support importing an image from storage into the document. To do it the package graphicx needs to be used.
	The image can be imported using includegraphics command.\\
	\textbf{\\Code:}
	\begin{lstlisting}
	\documentclass[12pt,a4paper]{article}
	\usepackage[left=1in,top=1in,bottom=1in,right=1in]{geometry}
	\usepackage{graphicx}\left( 
	\graphicspath{{./imag/}}
	\usepackage[rightcaption]{sidecap}
	\begin{document}
	A figure with fixed attribute:\\
	\begin{figure}[h]
	\caption{Example of a Triangle}
	\centering
	\includegraphics[height=3cm,width=3cm]{pics}\\
	\end{figure}
	Caption at below figure:\\
	\begin{figure}[h]
	\centering
	\includegraphics[height=3cm,width=3cm]{pics}\\
	\caption{Example of a Triangle}
	\end{figure}
	Caption right to the figure figure:\\
	\begin{SCfigure}[0.6][h]
	\caption{Example of triangle}
	\includegraphics[width=0.2\textwidth]{pics}
	\end{SCfigure}
	\end{document}
	
	\end{lstlisting}
	
	\textbf{\\Output:}\\
@thesis{ID,
	author = {author},
	title = {title},
	type = {type},
	institution = {institution},
	date = {date},
	OPTsubtitle = {subtitle},
	OPTtitleaddon = {titleaddon},
	OPTlanguage = {language},
	OPTnote = {note},
	OPTlocation = {location},
	OPTmonth = {month},
	OPTisbn = {isbn},
	OPTchapter = {chapter},
	OPTpages = {pages},
	OPTpagetotal = {pagetotal},
	OPTaddendum = {addendum},
	OPTpubstate = {pubstate},
	OPTdoi = {doi},
	OPTeprint = {eprint},
	OPTeprintclass = {eprintclass},
	OPTeprinttype = {eprinttype},
	OPTurl = {url},
	OPTurldate = {urldate},
}
		\includegraphics[height=10cm,width=7cm]{the}\\
	\textbf{\\Discussion:} Image with different types of size and caption is imported successfully.
\end{flushleft}

%fourthONE
\newpage
\begin{flushleft}
	
	
	\textbf{Experiment No:} 04\\
	\textbf{\\Experiment Name:} To display mathematical equation using LaTex.\\
	\textbf{\\Objective:}To learn how to display mathematical equation using LaTex.\\
	\textbf{\\Introduction:} 
	\begin{lstlisting}  
	To include equations in	LaTex \{equation} is used. syntax like 
	\sum, \frac can be used for using special symbols.
	\end{lstlisting}
	\textbf{\\Code:}
\begin{lstlisting}
\documentclass[12pt,a4paper]{article}
\usepackage[left=1in,top=1in,bottom=1in,right=1in]{geometry}
\usepackage{amsmath}

\begin{document}
Equation using inline\\
\\$f(x)=x^2+3x+2$ \\
$g(x)=\frac{1}{x^2}$\\
\\Equation using []
\[x^2=2y+1\]
Equation using align
\begin{align*}
f(x) &= x^2\\
g(x) &= \frac{1}{x}
\end{align*}

Equation with numbering
\begin{equation} \label{tp}
e^{\pi i} + 1 = 0\\
\end{equation}
The beautiful equation \ref{tp} is known as the Euler equation
\begin{equation}
g(x) = \frac{1}{x}
\end{equation}
Equation without numbering
\begin{equation*}
F(x) = \int^a_b \frac{1}{3}x^3
\end{equation*}
\end{document}
\end{lstlisting}

	\textbf{\\Output:}\\
		\includegraphics[height=8cm,width=7cm]{for}\\
	\textbf{\\Discussion:} Different types of equations are added successfully.
\end{flushleft}

%fiftheONE
\newpage
\begin{flushleft}
	
	
	\textbf{Experiment No:} 05\\
	\textbf{\\Experiment Name:} To display  matrix using LaTex.\\
	\textbf{\\Objective:} To know how to display  matrix using LaTex.\\
	\textbf{\\Introduction:} Matrix can be represent using \$ or equation command.\\
	\textbf{\\Code:}
	\begin{lstlisting}
	\documentclass[12pt,a4paper]{article}
	\usepackage[left=1in,top=1in,bottom=1in,right=1in]{geometry}
	\usepackage{amsmath}
	\begin{document}
	Matrix without parenthesis\\
	$$
	\begin{matrix} 
	a & b \\
	c & d 
	\end{matrix}
	$$
	
	Matrix with parenthesis:
	$$
	\begin{pmatrix} 
	a & b \\
	c & d 
	\end{pmatrix}
	$$
	
	$$
	\begin{bmatrix} 
	a & b \\
	c & d 
	\end{bmatrix}
	$$
	
	$$
	\begin{Bmatrix} 
	a & b \\
	c & d 
	\end{Bmatrix}
	$$
	
	$$
	\begin{vmatrix} 
	a & b \\
	c & d 
	\end{vmatrix}
	$$
	
	$$
	\begin{Vmatrix} 
	a & b \\
	c & d 
	\end{Vmatrix}
	$$
	\end{document}
	\end{lstlisting}
	
	\textbf{\\Output:}\\
		\includegraphics[height=10cm,width=6cm]{fiv}\\
	\textbf{\\Discussion:} Different types of matrix are added successfully.
\end{flushleft}

%sixthONE
\newpage
\begin{flushleft}
	
	
	\textbf{Experiment No:} 06\\
	\textbf{\\Experiment Name:} To demonstrate a cv with cover letter using LaTex.\\
	\textbf{\\Objective:} To know how to write a CV with cover letter using Latex.\\
	\textbf{\\Introduction:} A cover letter for your CV is an introductory message that accompanies your CV when applying for a job and a CV included all the information about a person.\\
	\textbf{\\Code:}
	\begin{lstlisting}
\documentclass[12pt, a4paper]{article}
\usepackage[top=1in,bottom=1in,left=1in,right=1in]{geometry}
\usepackage{graphicx}
\graphicspath{{./pics/}}
\usepackage{tcolorbox}
\begin{document}
Cody Fredrickson\\
\indent(123) 456-7891\\
\indent cfredrickson@email.com\\
\indent May 1, 2018\\

Dear Hiring Manager,\\

I am pleased to be applying for the Web Developer position at River Tech. My extensive experience with designing and developing websites using  Java and PHP on Rails matches well with your requirements.\\
\hline
In my previous experience at Cloud Clearwater, I developed several web-based applications for a project management software client. These cloud-based applications tracked and managed information technology project information for several of the world’s largest financial companies.\\
\hline
I am a proven team player, which is another important requirement for this position. I was recently voted by my peers to receive Cloud Clearwater’s Distinguished Developer Award for exhibiting ideal teamwork traits, including reliability, good communication, commitment, adaptability, and going above and beyond what is asked. Management also recognized my accomplishments in meeting project deliverable deadlines over 99\% of the time. I have included my CV here.\\
\hline
Thank you for taking the time to review my credentials and accomplishments. I look forward to answering your questions and learning more about this position and your development teams.\\
\hline
Sincerely,\\
\indent Mitu Akter




\newpage

\begin{flushright}
\includegraphics[height=4.5cm,width=3.5cm]{cvpic}
\end{flushright}
\vspace{-3cm}
\textbf{Mitu Akter}\\
kargaon,Katiadi\\Kishoregonj\\
Email: mitu16102001@gmail.com\\ 
Cell:017********\\

\begin{tcolorbox}
\textbf{Educational Informaiton}
\end{tcolorbox}
\hspace{-5mm}
\begin{tabular}{|p{2.3cm}|p{6.5cm}|p{1.5cm}|p{1cm}|p{2cm}|}
\hline
Examination& Inistitute& Subject& Year& GPA(Scale)\\
\hline
SSC &Kargaon Union High School  & Science& 2013 & 4.69(5.00)\\
\hline
HSC & Kishoregonj Govt. Mohila College  & Science& 2015 & 4.58(5.00)\\
\hline
B.Sc(Engn.) & Jatiya Kabi Kazi Nazrul Islam University& CSE & 2020 & 3.77(4.00)\\
\hline
\end{tabular}
\vspace{3mm}
\begin{tcolorbox}
\textbf{Fields of interest}
\end{tcolorbox}
\begin{itemize}
\item Data Mining
\item Machine learning 
\item Artificial intelligence
\end{itemize}
\begin{tcolorbox}
\textbf{Technical Abilities}
\end{tcolorbox}
\begin{itemize}
\item  \textbf{Languages:} C, C++, Java
\item \textbf{Database:} MySQL
\item\textbf{Script:} Javascript,Ajax,php
\item\textbf{Tools:}  LaTeX ,Matlab , Arduino, Autocad
\end{itemize}

\begin{tcolorbox}
\textbf{ Projects and seminars}
\end{tcolorbox}

\begin{itemize}
\item Seminar on Bioinformatrix  
\item Seminar and workshop on game design and development   
\item Seminar and workshop on arduino 
\end{itemize}

\begin{tcolorbox}
\textbf{Strengths}
\end{tcolorbox}
\begin{itemize}
\item Hard working, Positive Attitude, Social Interaction, Leader
\item \textbf{Languages}:Bengali, English
\end{itemize}
\begin{tcolorbox}
\textbf{Interest and hobbies}
\end{tcolorbox}
\begin{itemize}
\item Reading Books
\item Competitive programming
\item Traveling
\end{itemize}
\begin{tcolorbox}
\textbf{References}
\end{tcolorbox}
\begin{itemize}
\item Mr.Ahm Kamal,\\ Professor,\\Department of CSE\\ Jatiya Kabi Kazi Nazrul Islam Universit\\Email: mrX@email.com\\Cell: 019********
\item	Mst Jannatul Ferdous,\\Associate Professor,\\Department of CSE\\Jatiya Kabi Kazi Nazrul Islam University\\Email:mrsJ@email.com\\Cell: 017********
\end{itemize}

\end{document}
	\end{lstlisting}
	\textbf{Output:}\\
		\includegraphics[height=10cm,width=10cm]{cv1}\\
		\includegraphics[height=10cm,width=10cm]{cv2}\\
		\includegraphics[height=10cm,width=7cm]{cv3}\\
	\textbf{\\Discussion:} A CV with a cover letter is successfully created.
\end{flushleft}

%seventhONE
\newpage
\begin{flushleft}
	
	
	\textbf{Experiment No:} 07\\
	\textbf{\\Experiment Name:} To demonstrate a project report using LaTex.\\
	\textbf{\\Objective:}  To learn how to create  a project report using LaTex.\\
	\textbf{\\Introduction:} Project report consists of developments and overall progress of the process.\\
	\textbf{\\Code:}
	\begin{lstlisting}
	\documentclass[12pt,a4paper]{article}
	\usepackage[top=1in,bottom=1in,left=1in,right=1in]{geometry}
	\usepackage{biblatex}
	\begin{document}	
	\large Project Report On\\textbf{INTERNAL MARKS MANAGEMENT SYSTEM}\\
	\begin{flushleft}
	\textbf{Submitted by:}\\
	Tarana Nusrat\\                    
	Id:100000\\ 
	Session:2016\\                                                                
	\end{flushleft}	
	\begin{flushleft}
	\textbf{\\Under The Guidance of:}\\
	Habiba Sultana\\
	Lecturer\\
	Dept. Of CSE\\
	Jatiya Kabi Kazi Nazrul Islam University\\
	Trishal, Mymensingh\\
	\end{flushleft}
	Submission Date: 02/04/18 \\
	\thispagestyle{empty}
	\newpage
	\pagenumbering{roman}
	\tableofcontents
	\newpage
	\pagenumbering{arabic}
	\newpage
	\section*{ABSTRACT}
	The title of the project is Student Internal Marks Management
	System. The goal of this project is to create a system that will
	capture internal marks, calculate the sum of mid marks, attendance
	  marks.
	\newpage
	\section*{ACKNOWLEDGEMENT}
	Our project name is student internal marks management system.
	 We see ourselves as exceptionally fortunate and respected 
	 to have such a large number of brilliant individuals lead 
	 us through in finishing of this anticipate.
	\newpage
	\textbf{\huge Chapter 1}\\
	\section*{Introduction}
	The Internal marks assessment system helps teacher to reduce the overhead of
	 marks alculation and to manage them. Student Internal Marks Management 
	 System contains a  different type of information
	\section{Objectives:}
	\subsection{Key objective}
	\begin{itemize} 
	\item To provide an interface through which faculty can assist internal.
	\item To insert student name, roll, midterm marks, attendance.
	\end{itemize}
	\subsection{Main feathers}	
	\begin{itemize} 
	\item Add records
	\item Display record
	\end{itemize}
	\newpage	
	\textbf{\huge Chapter 2}\\
	\section{Advantages}
	\begin{itemize}
	\item This software is space and time efficient.
	\item It is small and user friendly.
	\end{itemize}
	\newpage	
	\textbf{\huge Chapter 3}\\
	\section{REQUIREMENTS}
	\subsection{SOFTWARE REQUIREMENT SPECIFICATION}
	\subsubsection{SOFTWARE REQUIREMENTS}
	\begin{itemize}
	\item Windows XP/VISTA/7/8/8.1/10
	\item Memory Space: Minimum 250 Mb
	\end{itemize}	
	\subsubsection{HARDWARE REQUIREMENTS} 
	\begin{itemize}
	\item  Keyboard
	\item Mouse
	\end{itemize}
	\subsection{SYSTEM ANALYSIS:}
	Systems analysis is a process of collecting factual data, understand
	 the processes involved, identifying problems and recommending feasible 
	 suggestions.
	\newpage	
	\textbf{\huge Chapter 4}\\
	\section{Result and Discussion}
	\subsection{Insertion}
	Insertion contain student name,roll,midterm marks,attendance.\\
	Figure 3: Insert information\\
	\subsection{Display}
	\subsection{Delete}
	\subsubsection{Deleting Process}
	\subsubsection{After Deletion}
	\subsection{Edit}
	\subsubsection{Editing Process}
	\subsubsection{After Edition}
	\newpage	
	\textbf{\huge Chapter 5}\\
	\section{Future scope}
	The current system is mainly designed to support the management of the 
	organization. In the future, the system can be enhanced to include details 
	of students who are failed. 
	\subsection{Limitations of the System:
	\begin{itemize}	
	\item Some features such as searching, sorting are not available in.
	\end{itemize}
	\newpage	
	\textbf{\huge Chapter 6}\\
	\section{Conclusion}
	It was an experience that changed the way we perceived project development.
	\newpage	
	.\begin{thebibliography}{1}
	\bibitem  hhttps://www.slideshare.net/SnehalVRaut/student-database-management-
	system-finale-23136675\
	\bibitem hhttps://www.scribd.com/document/359829335/26-Student-Management-
	System-Abstractdocx
	\end{thebibliography}
	\end{document}
	
	\end{lstlisting}
	\textbf{Output:}\\
	\includegraphics[width=8cm,height=8cm]{sv1}\\
	\includegraphics[width=10cm,height=10cm]{sv2}\\
	\includegraphics[width=10cm,height=3cm]{sv3}\\
	\includegraphics[width=10cm,height=3cm]{sv4}\\
	\includegraphics[width=10cm,height=8cm]{sv5}\\
	\includegraphics[width=10cm,height=3cm]{sv6}\\
	\includegraphics[width=10cm,height=8cm]{sv7}\\
		\includegraphics[width=10cm,height=7cm]{sv9}\\
	\includegraphics[width=10cm,height=4cm]{sv8}\\
								
	\textbf{Discussion:} We have successfully created a thesis paper with latex
\end{flushleft}

%eightthONE
\newpage
\begin{flushleft}
	
	
	\textbf{Experiment No:} 08\\
	\textbf{\\Experiment Name:} To demonstrate a lab report using LaTex.\\
	\textbf{\\Objective:} To Learn how to make a lab report using Latex.\\
	\textbf{\\Code:}
\begin{lstlisting}
\documentclass[12pt,a4paper]{article}
\usepackage[top=1in,bottom=1in,left=1in,right=1in]{geometry}
\usepackage{graphicx}
\usepackage{listings}
\begin{document}
\begin{center}
\LARGE\textbf{Jatiya Kabi Kazi Nazrul Islam University} \\
\Large\textbf{Trishal, Mymensing-2220}\\
\vspace{5mm}
\includegraphics[width=3.5cm,height=3cm]{un.jpg}
\end{center}

\begin{center}	
\large\textbf{Department Of Computer Science And Engineering}
\textbf{\\Course Name: }Microprocessor and Assembly Language Lab\\
\textbf{Course Code: }CSE-302\\	
{A Lab report on "Microprocessor and Assembly Language Lab"}\\
\end{center}
\vspace{2cm}
\begin{flushleft}
\textbf{Submitted  To:}\\
Pronab Kumar Mondal\\
Assistant Professor\\
Dept. of CSE\\
JKKNIU
\end{flushleft}
\begin{flushright}
\textbf{Submitted  by:}\\
Mitu Akter\\
Reg. No: 4853\\
Roll: 16102001\\
Session : 2015-16\\
Dept. of CSE\\
\end{flushright}
\newpage
\begin{flushleft}
\textbf{Experiment No:} 01\\
\textbf{\\Experiment Name:} An assembly program that can print a string. \\
\textbf{\\Objective:} To be able to print a string using LEA(Load effective address) in 8086 assembly language\\
\textbf{\\Code:}
\begin{lstlisting}
.MODEL SMALL
.STACK 100H
.DATA 
MSG DB 'Hello World! $'
.CODE
MAIN PROC
MOV AX,@DATA
MOV DS,AX
LEA DX,MSG
MOV AH,9
INT 21H	
MOV AH,4CH
INT 21H
MAIN ENDP
END MAIN
\textbf{Output:}\\
HELLO WORLD!
\textbf{\\Discussion:} The program code of printing string works successfully.\\

\end{lstlisting}

\textbf{Output:}\\
\includegraphics[height=12cm,width=10cm]{eit1}\\
\includegraphics[height=12cm,width=10cm]{eit2}\\
\textbf{\\Discussion:} A lab report is created successfully.\\
\end{flushleft}


	

%ninthoneONE
\newpage
\begin{flushleft}
	
	
	\textbf{Experiment No:} 09\\
	\textbf{\\Experiment Name:} To demonstrate a thesis paper using LaTex.\\
\textbf{\\Objective:} To learn to create a thesis paper using LaTex\\
	\textbf{\\Introduction:} Thesis report consists of a new theory or development of an existing one.\\
	\textbf{\\Code:}
\begin{lstlisting}
\documentclass[12pt,a4paper]{article}
\usepackage[utf8]{inputenc}
\usepackage[english]{babel}
\usepackage[document]{ragged2e}
\usepackage{graphicx}
\usepackage{biblatex}
\usepackage[toc,page]{appendix} 
\usepackage{blindtext}
\usepackage{titletoc}
\usepackage{tocloft,calc}
\addbibresource{sample.bib}
\usepackage{parskip}
\usepackage[margin=1in,left=1in,right=1in,includefoot]{geometry}
\begin{document}
\begin{titlepage}
\begin{center}
\huge
\textbf{HAND WRITING RECOGNITION USING NEURAL NETWORK}
\end{center}
\begin{figure}[h]
\centering
\includegraphics[scale=.4]{logo.jpg}
\end{figure}
\begin{center}
\vspace{.1cm}
\textbf {A DISSERTATION}\\
\vspace{.1cm}
\textbf {PRESENTED TO THE FACULTY OF SCIENCE AT THE 
UNIVERSITY OF JKKNIU}
\vspace{1cm}
\textbf {Major: Computer Science and Engineering}
\end{center}
\vspace{1cm}
\underline{ \textbf{SUBMITTED TO}}\\
\vspace{.5cm}
\textbf {KAZI MAHMUDUL HASAN MUNNA}\\
ID NO: 09206025\\
REGISTRATION NO: 09206025\\
SESSION: 2012-2013
\vspace{1cm}
\raggedright
\begin{flushright}
.............................\\
SIGNATURE
\end{flushright}
\underline{ \textbf{SUBMITTED BY}}\\
\vspace{.5cm}
\textbf {Fahim Abdullah(16102021)}\\
\textbf {Mitu Akter(16102001)}\\
SESSION: 2015-2016
\vspace{.5cm}
\begin{center}
\underline{\textbf { DEPARTMENT OF COMPUTER SCIENCE
AND ENGINEERING}}\\
\vspace{.1cm}
\textbf{JATIYA KABI KAZI NAZRUL ISLAM UNIVERSITY}\\
\textbf{BANGLADESH}
\end{center}
\end{titlepage}

\newpage
\pagenumbering{roman}
\tableofcontents
\section{Implementation of Neural Network}

\justify
One fundamental virtualization technology is machine
virtualization. It can create a large number of virtualmachines
(VM) on a given physicalmachine. Withmachine virtualization,
the cloud size can be increased by hundreds of times.

\subsection{Capacity for Networks}
\justifyThe second virtualization technology is network
virtualization. Network virtualization splits a physical
network into multiple small virtual networks. 
\subsubsection{Bandwidth: To Describe a Virtual Network}
\justify
Regarding bandwidth, my apartment, for example, uses 50
Mbps of xfinity Internet. The 50Mbps is the bandwidth of 
my Internet connection. 	

\subsection{Challenges to bandwidth estimation in virtual
networks} 
\justify
Virtual networks bring more challenges to bandwidth estimation
techniques. We explain various challenges to bandwidth
estimation in virtual networks below
\subsection{Improvements to bandwidth estimation in virtual
networks}
\justify
The contribution of this dissertation consists mainly of two
points.

\subsection{Organization}
\justify
We describe our work on token bucket shaper measurement in
Chapter 2. 
\newpage
\raggedright
\textbf{\huge Chapter 2}
\section*{\textbf {Networks}}
\line(1,0){450}
\section{Introduction}
\justify
Cloud computing is transforming a large part of IT industry, 
as evidenced by the increasing popularity of public cloud
computing services. 
\subsection{Networks with Token Bucket Shapers} 
\justify
In this section, we introduce tbf and tbf -like shapers, 
and discuss their impact on the dispersions
of a train of packets. 
\subsection{Tbf and tbf-like Shapers} \justify
shaper, if there are at least s bits of tokens available. are
consumed	from the token bucket. 
\subsection{Impact on Packet Dispersions}  
\justify
We use the following two simple networks to show the impact of
token bucket shapers ona train of packets.
\newpage
\raggedright
\textbf{\huge Chapter 3}
\section*{\textbf {Design and Challenges}}
\line(1,0){450}
(VM) than to buy server
\subsection{Goals} 
\justify
To answer this question,this chapter offers detailed information
about rate limiting in.
\subsection{Chapter Organization} 
\justify
We provide the background related to bandwidth allocation for a
VM and rate limiters in section 4.2.
\nwpage
\raggedright
\textbf{\huge Chapter 5}

\section{Conclusion and Future Work} 
\justify
As per our analysis in this dissertation, the two main problems
are rate limiters used in virtual
networks and incorrect time information used in
bandwidth-estimation software.
\subsection{Conclusion} 
\justify
In this dissertation, we studied bandwidth estimation in virtual
networks
\subsection{Future Work} 
\justify
With the rapid development of cloud technologies, there are now
many cloud vendors in
the industry; these include Amazon Web Services
\newpage
.\begin{thebibliography}{1}
\bibitem{latexcompanion}B.Admanson, Frank Mittelbach, and
Alexander Samarin. \textit{The \LaTeX\ Companion}.
Addison-Wesley, Reading, Massachusetts, 1993.
ibitem{knuthwebsite} Knuth: Computers and
Typesetting,
\\\texttt{http://www-cs-faculty.stanford.edu/\~{}uno/abcde.html}
\section{Appendix}
The performance analysis of MCU can be understood by the program
which is written into it.
\end{lstlisting}
	\textbf{Output:}\\
	\includegraphics[height=10cm,width=8cm]{s97.jpg}\\
	\includegraphics[height=10cm,width=12cm]{s98.jpg}\\
	\includegraphics[height=10cm,width=12cm]{s92.jpg}\\
	\includegraphics[height=10cm,width=12cm]{s91.jpg}\\
	\includegraphics[height=10cm,width=12cm]{s94.jpg}\\
	\includegraphics[height=10cm,width=12cm]{s93.jpg}\\
	\includegraphics[height=10cm,width=12cm]{s95.jpg}\\
	\includegraphics[height=3cm,width=12cm]{s96.jpg}\\

	\textbf{Discussion:} A thesis paper is created successfully.
\end{flushleft}

%tenthONE
\newpage
\begin{flushleft}
	
	
	\textbf{Experiment No:} 10\\
	\textbf{\\Experiment Name:} To demonstrate a project proposal using LaTex.\\
	\textbf{\\Objective:}To learn how to demonstrate a project proposal using LaTex\\
	\textbf{\\Code:}
	\begin{lstlisting}
	\documentclass[12pt,a4paper]{article}
	\usepackage[left=1in,top=1in,bottom=1in,right=1in]{geometry}
	\usepackage{graphicx}
	\begin{document}
	\begin{center}
	\LARGE\textbf{Jatiya Kabi Kazi Nazrul Islam University} \\
	\large\textbf{Department Of Computer Science And Engineering}\\
	\end{center}
	\large Project Proposal On\textbf{"Development of Phone Book"}\\
	Date of submission: 24-02-2019\\
	\begin{flushleft}
	\textbf{Submitted by:}\\
	Mst. Sabrina Sultana\\                    
	Id:17102045\\ 
	Session:2016-17\\                                                                
	\end{flushleft}
	\newpage
	\section*{Project title:} Development of Phone Book.
	\section*{Objective} 
	The objective of the software Phone Book is to provide a system 
	which manages the contact details of all the people and also
	 manages any occasion to be remembered as a remainder. 
	\section*{Motivation}\vspace{0.2cm}
	The motivation of Phone Book project are the following:
	\begin{itemize} 
	\item	Contact A new Contact is added where we are supposed
	to enter the name of the person, his contact details and e-mail
	 Id. Also the details can be edited if necessary.
	\item	Search The user can search any contact details either 
	through number or name. 
	\end{itemize}
	\section*{Introduction}
	The software Phone Book is to provide a system which manages 
	the contact details of all the people and also manages any occasion 
	 be remembered as a remainder. 
	\section*{Background state of the problem:}
	Firstly, a page of keypad shown through where we can enter the number. 
	\section* {Hardware Support:}
	\begin{tabular}{|p{6cm}||p{10cm}|}
	\hline
	Computer &	PC \\
	\hline
	Processor &	INTEL P3 \\
	\hline
	Processor Speed & 250 MHz to 833MHz \\
	\hline
	RAM & 128MB above \\
	\hline
	\end{tabular}
	\section*{Software Support}
	\begin{tabular}{|p{6cm}||p{10cm}|}
	\hline
	Web Server &  Apache Tomcat, Macromedia JRun,  \\
	\hline
	Programming Language & Java \\
	\hline
	\end{tabular}
	
	\section*{Outline of Methodology/Experiment Design}
	Firstly, a page of keypad shown through where we can enter the number.
	 As soon as we click on save.
	\section*{Conclusion}
	The users job of saving the contact details can be simplified through 
	this automated system.
	\end{document}
	\end{lstlisting}
	\textbf{Output:}\\
		\includegraphics[height=8cm,width=15cm]{p11}\\
	\includegraphics[height=10cm,width=15cm]{p22}\\
	\includegraphics[height=8cm,width=15cm]{p33}\\
	\textbf{Discussion:} A project proposal is created.
\end{flushleft}

%eleventhONE
\newpage
\begin{flushleft}
	
	
	\textbf{Experiment No:} 11\\
	\textbf{\\Experiment Name:} To demonstrate a slide using LaTex.\\
	\textbf{\\Objective:} To Learn how to make a slide using Latex.\\
	\textbf{\\Introduction:} Here we have used Beamer class and used Frankfurt theme.\\
	\textbf{\\Code:}
	\begin{lstlisting}
	\documentclass{beamer}
	\usetheme{Frankfurt}
	\usepackage{graphicx}
	\title{Title of the presentation}
	\author{author names}
	\date{dd mm yy}
	\begin{document}
	\maketitle
	\section{first page}
	\begin{frame}
	\frametitle{introduction}
	plain text
	content...
	\end{frame}
	\section{mid Page}
	\begin{frame}
	'\frametitle{items}
	\begin{itemize}
	\item one\pause[3]
	\item two
	content...
	\end{itemize}
	\end{frame}
	\section{last section}
	\begin{frame}
	\frametitle{explanation}
	\begin{columns}
	\begin{column}{0.5\textwidth}
	some text here \\image on the other side
	\end{column}
	\begin{column}{0.5\textwidth}  %%<--- here
	\begin{center}
	\includegraphics[width=0.5\textwidth]{e1_1.jpg}
	\end{center}
	\end{column}
	\end{columns}
	\end{frame}
	\end{document}
	\end{lstlisting}
	\textbf{Output:}\\
	\includegraphics[height=7cm,width=15cm]{s11.jpg}\\
	\includegraphics[height=7cm,width=15cm]{s22.jpg}\\
	\includegraphics[height=7cm,width=15cm]{s33.jpg}\\
	\textbf{\\Discussion:} A slide has been created using latex\\
\end{flushleft}




\end{document}